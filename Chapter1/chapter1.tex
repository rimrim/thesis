%*******************************************************************************
%*********************************** First Chapter *****************************
%*******************************************************************************

\chapter{Introduction}  %Title of the First Chapter

\ifpdf
    \graphicspath{{Chapter1/Figs/Raster/}{Chapter1/Figs/PDF/}{Chapter1/Figs/}}
\else
    \graphicspath{{Chapter1/Figs/Vector/}{Chapter1/Figs/}}
\fi

\section{User Authentication and Biometrics}
\label{sec:biometricIntro}
User authentication is the process of verifying the claimed identity of a user,
and is therefore a crucial component within the big picture of information and
network security. Generally speaking, there are three types of identifying modes
("factors"): Something You Know (SYK), such as password; Something You Have
(SYH), such as smartcard, and Something You Are (SYA), such as an iris or a
fingerprint idiosyncracies. The last factor, also known as biometrics, can be
considered to be the most convenient one, as it exempts system-users from
carrying or remembering something for authentication purposes.  Stimulated by
the reduced cost of the hardware it consumes and by its unsurpassed usability
level, biometric authentication has lately been deployed profusely on numerous
platform types.

Various body features have been proposed and used for recognition and
verification of a person's identity, the three most popular ones being fingerprints, face and
iris modalities. Some commercial applications enforce other recognition techniques, based on palm-prints, hand geometry and voice. There are many other idiosyncracies proposed by
researchers (such as gait, ear, keystroke dynamics, etc.), yet to attain a
sufficient level of technological maturity for deployment. Regardless of the
traits used, in biometrics verification systems, there are generally two stages:
enrollment and authentication. In the enrollment stage, a user registers his
biometric template on a server. In the authentication stage, he produces
a template afresh and submits it to the server to prove his identity. Unlike password
authentication, where a user always enters the same string he chose when
registering, in biometric systems, the query template of a genuine user is not exactly equal, but only very similar to the registered one, because, in the case of fingerprints, for instance, when
authenticating, the user might not touch the sensor on exactly the same corner he pressed while registering . The server has thus to compute some kind of distance between the
registered and the query templates, and compare the result to some threshold
values, in order to decide on the authentication results. Unlike SYK and SYH, the SYA
factor has two parameters, namely, False Acceptance Rate (FAR) and False
Rejection Rate (FRR): FAR is the rate at which the system wrongly accepts an
impersonating user as genuine, FRR is the rate at which the system rejects a genuine
user. Different biometrics authentication systems use different extraction
methods and distance measures to reduce FAR and FRR (this is also an active
research area). In this thesis, we do not focus on template extraction techniques
or on improving FAR and FRR; instead, we are interested in means providing extra
privacy and security features to any methods based on Hamming Distance
of biometrics represented by bitstrings.

\section{Privacy-preserving Biometric Authentication}
\label{sec:privacyLiteratures}
Many biometrics authentication systems lack protection for the data
stored on a server. A server breach incident can thus result in catastrophic
consequences: in 2015, nearly 6 millions plaintext fingerprint data tokens of US
government employees were leaked due to a security attack
\cite{OPMsays563:online}. When biometric data is revealed or stolen, the victims become vulnerable to impersonation attacks for the rest of their lives, as, unlike password or smartcards, one’s fingerprints or iris are nearly impossible to change. Biometric privacy poorly protected against server exposure can therefore be considered as the main drawback of SYA authentication. SYA proving nonetheless the most convenient authentication method from the usability point of view, a strong
motivation arose to enhance it with an adequate privacy protection of
biometric data. There are four main approaches to privacy-preserving biometrics
authentication at present (details available in surveys of \cite{jain201650} and
\cite{jain2008biometric}) \begin{itemize} \item Transforming biometric data
  using a key-less many-to-one transformation, also known as `Non-Invertible'
  transforms (e.g.~\cite{ratha2007generating}): Since the function is key-less,
  biometric impersonation security under server exposure is vulnerable to
  off-line brute-force attacks, also known as FAR
  attacks~\cite{uludag2004attacks,roberts2007biometric}, which are the biometric
  analogue of off-line dictionary attacks on weak password hashing. Moreover,
  the many-to-one mapping trades off the accuracy (FRR) performance of
  the biometric scheme. A solution to overcome such an issue is to introduce a
  cryptographic secret-key authentication factor, i.e., to transform the
  registered biometric template stored on the server by encrypting it with a
  secret key stored at the client's side. With this second authentication
  factor, verification will require the secret key, which prevents brute-force
  attacks on the biometric data. Generally speaking, two-factor security can still protect the system when either the secret key is exposed (e.g., the client's device is
  stolen), or the encrypted biometric data stored on the server is leaked (e.g.,
  server breach). All other approaches we consider below are based on this
  two-factor method.


\item Transforming biometric data using a keyed `distance-preserving' encryption scheme:  Here, the distance between the biometric query template and the biometric registered one is measured by the server using encrypted versions of them.  Typical techniques in this category include
            cancelable biometrics and biohashing (\cite{teoh2008cancellable}, \cite{jin2004biohashing}, 2P-MCC (\cite{cappelli2010minutia}). %The security of this approach is unclear that was not based on standard cryptographic assumptions.
However, the security of such encryption schemes is unclear, as they are based on heuristic and non-standard cryptographic assumptions, several of them being reputedly insecure (~\cite{lee2009inverse,lacharme2013preimage}).

%\item
%            depends on the security of the key, this is the main
%            limitations of the approach:
%        the user's privacy is not assured
%    as the server decrypts and read the biometrics data when doing
%authentication, or a malicious server can even brute-force the key and/or the biometrics itself (Hill-climbing attack).
\item Biometric cryptosystems / Fuzzy hashing: This approach is based on techniques using error-correcting codes (\cite{uludag2004biometric, nagar2010hybrid}) to extract a noiseless cryptographic key from noisy biometric data. However, depending on parameter settings, these techniques may leak significant information, and the trade-off between biometric accuracy (False Acceptance Rate and False Rejection Rate) and security remains controversial or not well understood. In practice, the implemented techniques require strong restrictions on accuracy: in particular, the underlying error-correcting codes are only capable of handling a specific range of the threshold, depending on the defined usability parameters.

\item Secure Computation / Homomorphic encryption techniques
    (\cite{yasuda2014practical}, \cite{shahandashti2012private},
    \cite{higo2015privacy}): In this approach, the biometric data is encrypted
    on the server using a homomorphic encryption scheme. The operations on
    encrypted data are meant to measure the similarity between the query and the
    stored templates. This approach has the potential to overcome the heuristic
    security issues of the above approaches (by using an encryption scheme based
    on standard cryptographic assumptions), as well as the biometric accuracy
    issues (by implementing homomorphically the same verification check as the
    underlying biometric scheme). However, existing protocols following this
    approach suffer from other significant drawbacks. In particular, earlier
    approaches~\cite{shahandashti2012private} were based on the Paillier
    \emph{additive} homomorphic scheme and are granted with limited efficiency due to
    both the inefficiency of homomorphic operations of the Pailler scheme, and
    the extra protocol overhead required to handle the `addition-only'
    homomorphism limitation of the this scheme. Besides, the Paillier system
    does not provide \emph{long term} security (it is not quantum resistant).
    Its efficiency has been
    significantly improved by a recent protocol due to Yasuda et
    al.~\cite{yasuda2014practical}, based on lattice-based SomeWhat Homomorphic
    encryption (SWHE), which supports both homomorphic additions and multiplications.
    The cryptosystem used is believed to be quantum-resistant as
    well.
    However, this protocol requires the use of a trusted-third party server to
    verify the final authentication result: The main server stores encrypted
    biometric data and computes HD homomorphically, while another other server stores the
    decryption key and decrypts/verifies the HD.
    The method is thus not secure against
    server exposure attacks, as keys hold by the trusted server are able to decrypt data stored on the
    main one.
  \end{itemize}

  We focus our work on the fourth approach, adding various improvements to it, the goal
  being to design an authentication protocol able to satisfy more complex security and
  usability requirements according to the current trends. These requirements include:
\begin{description}
\item [Privacy against server exposure.] The biometric data should be securely
  encrypted prior to uploading it to the server, using a key known only by the
  biometric data owner. Ideally, the server storing the encrypted biometric information should
  not be able to comprehend the data. In other words, it should not possess the
  encryption key.

\item [Quantum resistant privacy. ] Considering that quantum computing is developing
  fast and that biometric data will persist over the lifetime of a user,
  encrypted biometric data should be granted long term security against quantum
  computing attacks.
\item [No reliance on trusted third parties. ] With the wide exposure of cloud
  computing to potential attacks, the protocol should not rely on any
  cloud-based trusted third parties to carry out the authentication process.
\item [Security against malicious clients.] An attacker attempting to impersonate
  the real client before the server should not be able to authenticate without
  genuine biometric features, even being malicious and not following the
  authentication protocol. Consequently, an impersonation attacker should never be assumed to
  be `honest but curious'(HBC).
\item [Two factor security.] If the client is responsible for decrypting
  biometric-related data, the protocol should be multi-factor secure: an
  attacker with a compromised key should not be able to authenticate without
  genuine biometric features.
\item [Practical performance.] The computation time and communication size of
  the whole protocol should remain within a practical time frame.
\end{description}
Previous protocols in the literature do not meet one or more of our
requirements. In particular, many previous protocols
(\cite{bringer2007application}, \cite{erkin2009privacy},
\cite{osadchy2010scifi}) assume honest-but-curious clients and are insecure in
an authentication context involving malicious clients. Some protocols
involving malicious clients (\cite{shahandashti2012private},
\cite{vsedvenka2015secure}) are not quantum-resistant. Previous practical
quantum-resistant protocols (\cite{yasuda2014practical},
\cite{mandal2015comprehensive}) are not secure against malicious clients and
involve a trusted third-party verification server, thus compromising the
client's privacy.

\section{Contributions of the thesis}
\label{sec:thesisContributions}
As has been outlined, research gaps result from the combination of non-satisfied
security constraints stated in Section 1.2. Either existing protocols are not
safe against authentication involving malicious clients, or, when they are, they
do not prove quantum-resistant. Previous quantum-resistant protocols not only
involve a third party, but lack robustness with respect to malicious clients
security model. The aim of our work is to overcome these security issues while
providing satisfactory practical performance, in terms of computation time and
communication size.  We propose different protocols to support the above stated
requirements. Our technique combines state-of-the-art cryptographic tools, such
as Homomorphic Encryption (HE) and Zero-Knowledge-Proof (ZKP), aimed at
balancing the security and the usability of the system. The techniques we put
forward are all lattice-based, which appears to be one of the best framework for
preserving long term security against quantum attacks. Our protocol also
achieves privacy protection against server exposure, by avoiding to rely on a
trusted third party. Our contributions include:
\begin{itemize}
\item Four different Quantum-resistant and provable-secure biometric authentication protocols
  that do not rely on trusted third-parties. The server stores the encrypted data and performs
  homomorphic operations to compute the distance (Hamming Distance) allowing to decide on the
  authentication result. It does not need a third party to decrypt the encrypted
  HD but, instead, sends it to the client for decryption. The client uses ZKP to
  convince the server that the ciphertext was correctly decrypted.
\item The protocols provide security under the malicious client model.This is achieved
  by new ZKP techniques we designed specifically for different ciphertext
  packing methods. It is also applicable to the proving of plaintext knowledge for
  the BGV Homomorphic Encryption scheme (\cite{brakerski2011fully}) we
  adapted. We apply different ZKP variants based on \cite{stern1993new} and
  \cite{schnorr1989efficient}, the two main approaches for Zero
  Knowledge Proof.
\item Due to the noise inherent to lattice-based homomorphic encryption and
  its correlation with the evaluated ciphertext, we observe that there can be
  information leakage affecting the original plaintexts used in the homomorphic
  computations of a two-factor attacker having exposed the client's secret key. We
  propose an approach to cover such leakages without significantly reducing the
  efficiency of the protocol: the approach is a new application of Renyi
  Divergence (RD)-based analysis to uncover the security of the protocol with a
  small 'imperfect' one-time pad. The correlation of
  Homomorphic Encryption noise with the original plaintext before homomorphic
  evaluation, has been considered to be a problem of "circuit privacy" in theoretical HE
  literature (\cite{sander1999non}, \cite{ishai2007evaluating}), but the
  proposed solutions (\cite{homenc}, \cite{ostrovsky2014maliciously},
  \cite{gentry2010hop}) involve 'smudging' (imperfect masking) or bootstrapping
  techniques. Such techniques produce an exponentially large noise (in the security parameter), which
  reduces efficiency. By contrast, our Renyi-based method can manage with much smaller
  imperfect masks, which leads to an increased efficiency of the process. To our knowledge, this is the first application of
  Renyi divergence techniques to circuit privacy of HE.
\item We propose a new Oblivious Transfer (OT) approach to obtain inputs to
  Garbled Circuits under the malicious client model (the inputs' correctness need to be proved that they match the Homomorphic Encryption plaintext format). Unlike traditional OT approaches, where the receiver obtains
  keys as inputs to a garbled circuit, our one is compatible with the
  underlying homomorphic cryptosystem and smoothly integrates the encryptions of the
  keys to the circuit. The approach proves effective with just one level of
  homomorphic multiplication operations and can be deployed on various application
  scenarios requiring secure multi-party computations.
\end{itemize}

\section{Organization of the thesis and related publication}
The thesis is organized as follows:
\begin{description}
\item In Chapter \ref{chap:definitions}, we describe the general notation and
  recall the basics of Lattice-based cryptography, Authentication, Zero
  Knowledge Proof and Secure Multi-party Protocols used throughout the
  thesis.
\item In Chapter \ref{chap:firstProtocol}, we present our first variant of the
  protocol removing the role of trusted third parties in a privacy
  preserving authentication system.
\item In Chapter \ref{chap:renyiDivergence}, we present the new technique to
  cover circuit privacy while still keeping the parameters of the homomorphic
  cryptosystems within practical thresholds.
\item In Chapter \ref{chap:thirdProtocol}, we construct the third variant of the
  protocol for both computing and comparing the Hamming Distance
  homomorphically (Improving the privacy of the previous protocols which only compute HD on ciphertexts).
\item In Chapter \ref{chap:fourthProtocol}, we present the last variant of the
  protocol, which removes the computation and communication overheads (and the
  extra cost they imply) during the initialization stage at the cost of an extra one-time precomputation and storage.
\end{description}
\label{sec:introOrganize}

In the last Chapter, we conclude the thesis and state the open problems for
future research. The results of Chapter \ref{chap:firstProtocol} and
\ref{chap:renyiDivergence} are contained in the following publication. 

Dinh, T., Steinfeld, R., \& Bhattacharjee, N. (2017, December). A Lattice-Based Approach to Privacy-Preserving Biometric Authentication Without Relying on Trusted Third Parties. In Information Security Practice and Experience - 13th International Conference, ISPEC 2017, Melbourne, VIC, Australia, December 13-15, 2017, Proceedings. Lecture Notes in Computer Science 10701, Springer 2017.


%%% Local Variables:
%%% mode: latex
%%% TeX-master: "../thesis"
%%% End:
