\chapter{The First Protocol - Computing HD Homomorphically}
\label{chap:firstProtocol}

% **************************** Define Graphics Path **************************
\ifpdf
    \graphicspath{{Chapter3/Figs/Raster/}{Chapter3/Figs/PDF/}{Chapter3/Figs/}}
\else
    \graphicspath{{Chapter3/Figs/Vector/}{Chapter3/Figs/}}
\fi

\section{Introduction}
\label{sec:firstProcIntro}
Many biometrics privacy preserving authentication protocols rely on a trusted
third party to keep the client's secret key used to decrypt the authentication
result \missref{}. Our first attempt to improve the security of such schemes has
been to remove the role of such a party. As it proves inconvenient, due to
privacy issues, to allow a third party server to keep the client's secret key, the
distance plaintext decryption should be performed by the client, which, in turn,
raises the following challenges:
\begin{description}
\item[Multifactor security] A malicious client with a compromised secret key
  about the distance between the registred template and the extracted template while he decrypting the homomorphic ciphertext result sent by the server.
\item[Malicious Client Model Impact] It is important for the protocol to be secure
  against the \textit{active client }model, by which an adversary does not simply follow the
  protocol transcript but also tries to draw information from it (this model is also known
  as \textit{Semi Honest}, \textit{Honest-But-Curious}, or \textit{Passive} Client Model). 
\end{description}

The first issue has been solved forcing the server to mask the distance between templates by means of some
random value before sending it to the client
(\cite{mandal2015comprehensive}). In this chapter, we discuss the solution to
the second issue, where a new Zero-Knowledge-Proof (ZKP) technique for
lattice-based cryptosystems is introduced so as to compel the client to prove that the protocol is followed honestly. The technique not only provides a
proof of plaintext knowledge, it has another important advantage, namely, to prove the
binary format of the query template as well. Therefore, we can use the ZKP at
two stages: When the client first sends the query in order to start authenticating,
and at the final step, proving that the Hamming Distance is correctly decrypted.


\section{Related Work}
\label{sec:iscRelatedWor}
In a biometric authentication system, a user $\mathcal{U}$ first registers his fingerprint template $X$ on the server
$\mathcal{S}$. $\mathcal{U}$ later authenticates before $\mathcal{S}$ using the same finger with a template $Y$, $\server$
uses an algorithm $Verify(X,Y)$ to obtain the result of the authentication: \textbf{Accept} or
\textbf{Reject}. Different biometric systems might use different methods to compute the distance $\Delta$ between $X$ and
$Y$ within the algorithm $verify$. The distance $\Delta$ is compared to some predifined threshold value $\tau$ to determine
the result of the authentication trial. We refer the reader to \cite{jain2007handbook} for biometric feature extraction and
comparison techniques. Unlike password based systems, where $\user$ always uses one and the same same query for many authentication trials,
all biometric systems implement the concept of False Acceptance Rate (FAR), corresponding to the system's \textbf{Accept} answer while receiving an incorrect
template; and False Rejection Rate (FRR), where the system's \textbf{Reject} answer follows reception of a genuine one.  Balancing these 2 rates
while keeping good performances is one of the main challenges that fingerprint verification algorithms
\cite{FVConGoi2:online} try to solve. In this work, we consider that biometric data are represented as binary
codes and that HD is used to measure the similarity between two of them. We refer the readers to \cite{daugman2003importance}
or \cite{FujitsuD7:online} for examples of 2048-bit iris code generation and HD comparison.

There are three main approaches for privacy-preserving biometric authentication (\cite{jain201650},
\cite{belguechi2011overview}, \cite{jain2008biometric}). In the \emph{Feature transformation approach} (cancelable
biometrics or biohashing, such as \cite{teoh2008cancellable}, \cite{cappelli2010minutia}), the template data are
encrypted using a client's key. Being a single factor protection, it has a low security level as the key might be leaked. The \emph{Biometric
  cryptosystem approach} (fuzzy vault and fuzzy commitment, \cite{uludag2004biometric}, \cite{nagar2010hybrid}) is based
on error correcting codes and the trade-off between biometric accuracy and security seems to be insufficiently understood . We focus
our work on the last approach, \emph{Homomorphic Encryption}, which appears as the best candidate to provide all of the system design requirements mentioned.

The idea was first proposed in 2006 (\cite{schoenmakers2006efficient}) using the Paillier addictive homomorphic system
(\cite{paillier1999public}). In 2010, Osadchy et al. (\cite{osadchy2010scifi}) provided a privacy-preserving feature by
combining Paillier system with an oblivious transfer protocol. SCiFI uses 900-bit vector to represent face image data and
Hamming Distance (HD) to compare two vectors. \cite{blanton2011secure} developed a similar system for iris and
fingerprints but using the DGK cryptosystem \cite{damgard2008homomorphic} and garble circuit technique instead. They
represented biometric data as 2048-bit vectors and also used HD for threshold comparison. \cite{yasuda2014practical}
proposed an approach based on Somewhat Homomorphic Encryption (SHE) \cite{brakerski2011fully}. They introduced a
ciphertext packing technique to speed up the HD computation operation. There have been variations and improvements over
time (\cite{shahandashti2012private}, \cite{mandal2015comprehensive}, etc.). However, most of the protocols are only
secure against a semi-honest client, many of them relying on one or more trusted third parties with the client's secret key being used to decrypt the HD.



\section{The BV Cryptosystem and Yasuda Packing Method}
\label{sec:firstProcBGV}
Although the idea of Fully Homomorphic Encryption (arbitrary number of
operations) is feasible, its performance has not been considered practical
enough. We discussed some generic LWE based cryptosystems and their properties
(Section \ref{sub:LWE} ), in this chapter, we look specifically at BV
cryptosystem, which is a Somewhat Homomorphic Encryption system that allows
addition and some levels of multiplication (\cite{brakerski2011fully}). In later
chapters of the thesis, an extension of this cryptosystem (namely, BGV
\cite{brakerski2014leveled}) and its variant is also used frequently. We discuss
the basic operation of the BV system in this chapter.  The BV cryptosystem works
as follows.
\begin{description}
\item[Setup.] Initiate $(n,m,q,t, \chi)$ to define the ciphertext space $R_q$,
  the plaintext space $R_t = \frac{\mathbb{Z}_{t}[x]}{x^{n} + 1}$, and the error
  distribution, note that \(t << q\).
\item[KeyGen.] The secret key $sk$ can be chosen by selecting a small element
  $\mathbf{s} \in R_{q}$, $\mathbf{s} \randomsample \chi^{n}$ can be done for
  this step. The public key $pk$ is a pair of ring elements
  $(\mathbf{p_0},\mathbf{p_1})$, where $\mathbf{p_1} \randomsample R_q$ and
  $\mathbf{p_0} = -(\mathbf{p_1}\mathbf{s} + t\mathbf{e})$ with
  $\mathbf{e} \randomsample \chi^{n}$.
\item[Encryption.] Given a plaintext $\mathbf{m} \in R_t$ and a public key
  $pk=(\mathbf{p_0},\mathbf{p_1})$, the encryption first calculates
  $\mathbf{u},\mathbf{f},\mathbf{g} \randomsample \chi$, then computes a fresh
  ciphertext by
    \[
      Enc_{pk}(\mathbf{m}) = (\mathbf{c_0},\mathbf{c_1}) =
      (\mathbf{p_0}\mathbf{u} + t\mathbf{g} + \mathbf{m},
      \mathbf{p_1}\mathbf{u} + t\mathbf{f})
    \]
    By convention, $\enc{P}$ denotes the encryption of a plaintext $P$ under the
    BV scheme with the public key, and randomness is not taken into
    account. When the noise used in the encryption has to be specified,
    $\enc{(P,e)}$ is used to represent it, $e$ being the noise.
  \item[Decryption.] Although the above encryption generates ciphertexts of 2
    elements only in $R_q$, the homomorphic operations (discussed next) will
    make the ciphertext longer. The decryption of the ciphertext appears as
    $c=(\mathbf{c_0},\mathbf{c_1},\dots,\mathbf{c_L})$, with secret key
    $sk = (1, \mathbf{s}, \mathbf{s^2},\dots, \mathbf{s^L})$ as
    $ Dec(\mathbf{c},sk) = \left[\left[ \langle \mathbf{c}, \mathbf{sk} \rangle
      \right]_Q \right]_t $.
  \item[Homomorphic Operations.] Given 2 ciphertexts
    $c = (\mathbf{c_0},\mathbf{c_1},\dots,\mathbf{c_L})$ and
    $c' = (\mathbf{c'_0},\mathbf{c'_1},\dots,\mathbf{c'_{K}})$, before any
    operations are carried out, the ciphertexts are padded with zeros first if
    necessary (to make $K = L$, assuming that, initially, $K < L$).  The
    homomorphic addition $add(c,c')$ is computed by the component wise addition
    $add(c,c') = (\mathbf{c_0} +\mathbf{c'_0}, \dots,
    \mathbf{c_L}+\mathbf{c'_L})$. The homomorphic multiplication $mult(c,c')$ is
    computed by
    $mult(c,c') = (\mathbf{\hat{c_0}}, \mathbf{\hat{c_1}}, \dots,
    \mathbf{\hat{c}_{2L-2}})$ with
    $ \sum_{i=0}^{2L-2}\mathbf{\hat{c}_i}z^i = \sum_{i=0}^{L-1}\mathbf{c_i}z^i
    \times \sum_{j=0}^{L-1}\mathbf{c'_j}z^j $, where $z$ denotes a symbolic
    variable.
\end{description}

\subsection{Ciphertext packing}
\label{sub:ciphertext_packing}
Given a bit string plaintext $m \in \{0,1\}^*$, there are several ways to encode
it as a polynomial, or a ring element \textbf{m} $\in$ R$_{t}$ before
encryption. A recent popular approach for BV cryptosystems is called CRT packing
method, and based on the Chinese Remainder Theorem \cite{smart2014fully}. The
method allows Single Instruction, Multiple Data (SIMD) operations on encrypted
data. However, we do not use this packing technique in this chapter, later in
chapter \ref{chap:fourthProtocol} we propose another approach for such packing
method. This chapter focus on the method of \cite{yasuda2014practical}, which is
an extension of \cite{naehrig2011can}, the technique allowing HD computation
with just one level of multiplication.

Biometrics are used in authentication as the human features they record
systematically differ from one person to another. There are many algorithms that
extract the data used in the process (\cite{FVConGoi2:online}) (we refer to this
data as "biometric templates"). Many works (such as \cite{daugman2003importance}
and \cite{FujitsuD7:online}) use Hamming Distance (HD) as the metrics to measure
the similarity of the stored and query templates, and we equally do so to
evaluate our hypothesis (\ref{protocol:step3}). In the work of
\cite{yasuda2014practical}, the authors propose an innovative to compute the HD
of two n-bits templates $X$ and $Y$ in the ciphertext domain. Specifically, the
method packs all the bits of a template into a single ciphertext, and a linear
combination of homomorphic operations would purveys an HD result. We sketchily
present their method below.
\begin{definition}
  For $\mathbf{T} = (t_0, \dots, t_{n-1}$ and
  $\mathbf{Q} = (q_0, \dots, q_{n-1})$, we define two types of polynomials in
  the ring $R_q$ of the SHE schme:
  \[
    pm_1(\mathbf{T}) = \sum_{i=0}^{n-1}t_ix^i \ \textnormal{and} \
    pm_2(\mathbf{Q}) = - \sum_{j=0}^{n-1}q_jx^{n-j}
  \]
  The two types of packed ciphertexts are defined as
  \[
    ct_{1}(\mathbf{T}) = Enc(pm_1(\mathbf{T})) \ \textnormal{and} \
    ct_{2}(\mathbf{Q}) = Enc(pm_2(\mathbf{Q}))
  \]
\end{definition}
The main idea of \cite{yasuda2014practical} is that, if in the ring $R_q$ where
$x^n = -1$, then, when multiplying $pm_1(\mathbf{T})$ by $pm_2(\mathbf{Q})$, the
constant term of the result will be the inner product
$\langle \mathbf{T}, \mathbf{Q}\rangle$. If homomorphic multiplication is
applied to the ciphertexts, the ciphertext of their inner product will similarly
show as a result of it. Furthermore, this result can be used to compute HD as
shown below, the operation costs being one level of multiplication with 3
additions and 3 multiplications on ciphertexts.

\begin{theorem}
  Let $C_1 = - \sum_{i=0}^{n-1}x^{n-i}$ and
  $C_2 = 2 - C_1 = \sum_{i=0}^{n-1}x^i$. Let $Enc(HD)$ be a ciphertext given by
  \[
    ct_1(\mathbf{T})*Enc(C_1) + ct_2(\mathbf{Q})* Enc(C_2) -
    2*ct_1(\mathbf{T})*ct_2(\mathbf{Q})
  \]
  Then, the constant term of $Dec(Enc(HD))$ gives the Hamming Distance of
  $\mathbf{T}$ and $\mathbf{Q}$.
\end{theorem}

\begin{algorithm}
\caption{HD Computation Homomorphically}\label{alg:EvalDistance}
\begin{algorithmic}[1]
\Procedure{EvalDistance}{$\mathbf{T,Q}$}
\State $C_1 \gets - \sum_{i=0}^{n-1}x^{n-i}$
\State $C_2 \gets 2 - C_1$
\State $C_{HD} \gets ct_1(\mathbf{T})*Enc(C_1) + ct_2(\mathbf{Q})*
Enc(C_2) - 2*ct_1(\mathbf{T})*ct_2(\mathbf{Q})$
\State \textbf{return} $C_{HD}$
\EndProcedure
\end{algorithmic}
\end{algorithm}

\section{Zero-knowledge proof system}
\label{sec:introzkp}
Suppose that \emph{Paul} has found a solution
to a hard problem and he wants to convince \emph{Vicky} that the problem has
been solved by him. \emph{Paul} can simply write out the solution and give it
to \emph{Vicky}. However, \emph{Vicky} could be as
dishonest as to show the solution to others and claim that
it was her who solved the problem. The Zero Knowledge Proof (ZKP) is a
beautiful cryptographic concept allowing \emph{Paul} to produce the proof
in such a way that \emph{Vicky} does not learn any information besides the fact that the problem has been solved. Since introduced
in the 80s \cite{goldwasser1989knowledge}, ZKP has been extensively
studied and is currently an important building block of many cryptographic
protocols: identification schemes ( \cite{fiat1986prove}, \cite{feige1988zero},
\cite{guillou1990paradoxical}, \cite{schnorr1991efficient},
\cite{stern1993new}, \cite{kawachi2008concurrently},\dots), group signatures(
\cite{camenisch1997efficient}, \cite{ateniese2000practical},
\cite{boneh2004short}, \cite{boneh2004group}, \cite{groth2007fully}\dots),
anonymous credential systems ( \cite{camenisch2008efficient},
\cite{camenisch2001efficient}, \cite{belenkiy2009randomizable},
\cite{chase2013malleable},\dots), interactive encryption protocols(
\cite{galil1985symmetric}, \cite{goldreich2009foundations},
\cite{goldwasser2005proof}, \cite{katz2003efficient},\dots), and many more.



% At a simplified high level, the system works by evaluating the Hamming Distance (HD) between the encrypted biometric
% query and encrypted registered biometric homomorphically on the server using the SWHE, sending the encrypted HD back to
% the client for decryption and letting the client send back the decrypted HD to the server for comparing with a threshold
% . To secure the protocol against a malicious client, we add appropriate
% Zero-Knowledge (ZK) Proofs of Knowledge by the client to convince the server that the client has done the correct
% computations. However, achieving this goal efficiently is not trivial for the following reasons.

% First, for efficiency, the Yasuda protocol needs to use plaintext packing techniques (\cite{yasuda2014practical}), and we need to design new ZK
% proofs for the approporiate relations to support this packing. As our first technical contribution, we show how to adapt
% the Stern ZK proofs to obtain suitable ZK proofs for our protocol. (add more info?)

% Second, due to the noise inherent inside lattice-based homomorphic encryption and its correlation with the
% homomorphicaly evaluated plaintexts (the registered template) it turns out the above simplified version of our protocol
% may leak information on the registered template to a two-factor malicious impersonating client that has access to the
% victim client's secret key, i.e. we observe that is not enough for the server to mask the HD plaintext with a random
% one-time-pad mask before sending the ciphertext to the client (as done also in [Yasuda]; the noise in the ciphertext
% must also be masked. Unfortunately, a perfect one-time-pad mask cannot be used for the noise as it must be `small' for
% decryption. A similar problem has been observed in theoretical studies of circuit privacy of lattice-based secure
% computation protocols~\cite{circuit privacy}, but those solutions yield inefficient protocols. As our second technical
% contribution, we show a new application of Renyi divergence based analysis~cite{RD} to show the malicious security of
% our protocol with a small `imperfect' one-time pad. To do so efficiently, we show how to extend previous technical
% bounds on Renyi divergence between a discrete Gaussian and its offset in for Renyi divergence of order 2 to the case of
% Renyi divergene of order infinity.

\section{Stern-based variant ZKP}
\label{sec:firstProcZKP}
\subsection{Stern-based ZKP}
\label{sec:Stern-basedZKP}
Recall that we need to construct a proof for the ISIS relation:
\[
  R_{ISIS_{n,mq,\beta}} = \{ ((\mathbf{A},\vec{y}),\vec{x})
  \in \mathbb{Z}_q^{n\times m} \times \mathbb{Z}_q^n
  \times \mathbb{Z}^m: (\|\vec{x}\|_\infty \leq \beta) \land
  (\mathbf{A}\vec{x} = \vec{y} \mod q) \}
\]
There are several approaches to construct such a proof (e.g. \cite{lyubashevsky2008lattice},
\cite{micciancio2003statistical}, \cite{stern1993new}). We here discuss an approach based on \cite{stern1993new}. We refer our readers to the original paper for the detailed steps of the protocol and denote the whole proof protocol
\(\mathbf{Stern}_{A,x,y}\). It can be applied to implement a Zero Knowledge Proof of
Plaintext Knowledge (ZKPoPK) for latticed-based cryptosystems. For example, an extension of the work of \cite{ling2013improved}  (denoted by \(\mathbf{SternExt}\)), allows to compute plaintext knowledge by proving the encryption relation
of Regev's cryptosystem \cite{regev2009lattices}:

\begin{equation*}
  \label{eq:regev_zkp}
  \begin{multlined}[][\LTcapwidth]
    R_{Regev}^{q,m,n,t,\chi} = \{ ((p_0,p_1),(c_0,c_1),\vec{r}||M) \in (\mathbb{Z}_q^{m \times n} \times
    \mathbb{Z}_q^{m})
    \times (\mathbb{Z}_q^n \times \mathbb{Z}_q) \times \mathbb{Z}_q^{n+1} : \\
    (c_0 = p_0\vec{r}) \land (c_1 = p_1\vec{r} + M) \}
  \end{multlined}
\end{equation*}

The proof works by letting $\mathbf{A'} = \begin{bmatrix} p_1, 1\\p0, 0
\end{bmatrix} $ and $\mathbf{y} = \begin{bmatrix} c_1\\c_0
\end{bmatrix}$ be the public parameters and by letting $\mathbf{x} =
\begin{bmatrix}
  \vec{r}\\M
\end{bmatrix}$ be the \emph{Prover}'s witness. We observe that $\mathbf{A'} \mathbf{x} = \mathbf{y} \mod q$, that is,
$\mathbf{x}$, is a solution to the ISIS problem, provided that $\norminf{\vec{r}} \approx |M|$ AND the \emph{Prover} within a symmetric key setting. In many other contexts, the client does not know
$\vec{r}$, as encryption is ensured by other parties using public keys. For such situations, we can look at the decryption
equation:
\begin{align*}
  c_1 - c_0\vec{s} &= p_1\vec{r} + M - p_0\vec{r}\vec{s} \\
                   &= \mathbf{A}\vec{s}\vec{r} + t\vec{e}\vec{r} + M - \mathbf{A}\vec{r}\vec{s}\\
                   &= t\tilde{e} + M
\end{align*}
Therefore, we can write out the decryption relation as:
\begin{equation*}
  \begin{multlined}[][\LTcapwidth]
    R_{Regev,dec}^{q,m,n,t\chi} = \{((p_0,p_1),(c_0,c_1),\vec{s},\vec{e},\tilde{e}, M) \in (\mathbb{Z}_q^{m \times n}
    \times \mathbb{Z}_q^{m})\times (\mathbb{Z}_q^n \times \mathbb{Z}_q)\times \chi^n \times \chi^n \times \chi \times
    \zzq:\\ (p_1 = p_0\vec{s} + t\vec{e}) \land (c_{1} = c_{0}\vec{s} + t\tilde{e} + M) \}
  \end{multlined}
\end{equation*}
Similarly, we can let $\mathbf{A}_{Stern} = \begin{bmatrix} c_0, t, 0, 1\\p_0, 0, t, 0
\end{bmatrix}$ (the public parameters) and let \(\mathbf{X}_{Stern} = [\vec{s},\tilde{e},\vec{e},M]^T\) and try applying \textbf{SternExt} to obtain the ZKPoPK. However, this will not work because the original
solution proves that $\norminf{\mathbf{X}_{Stern}} <
\beta$. In such cases, we want to prove the bound of separate components differently: in our above example,
$\norminf{\tilde{e}} >
\norminf{\vec{e}}$. In a Regev cryptosystem, the problem might not be clear enough, as the norms of each vector in
$\mathbf{X}_{Stern}$ are quite close to each other. In other latticed-based system, the difference can be big, as for example, the BV system's decryption relation:
\begin{equation*}
  \begin{multlined}[][\LTcapwidth]
  R_{BV}^{q,n,t,\chi} = \{((\mathbf{c_0}, \mathbf{c_1}),(\mathbf{p_0}, \mathbf{p_1}), \mathbf{s},\mathbf{e'}, \mathbf{e}, \mathbf{m} \in
  (\rrq \times \rrq) \times (\rrq \times \rrq)  \times \chi^n \times \chi^n \times \chi^n \times R_t:\\
  (\mathbf{p_1}\mathbf{s} + t\mathbf{e} = -\mathbf{p_0} ) \land (\mathbf{c_1}\mathbf{s} - t\mathbf{e'} -\mathbf{m} = -\mathbf{c_0})
  \}
  \end{multlined}
\end{equation*}

Our \emph{Prover}'s witness in this situation is
\(\mathbf{X}_{Stern} = [\mathbf{s, e', e, m}]^{T}\) with $\norminf{s} \approx \norminf{e} < \norminf{e'} << \norminf{m}$.  Therefore, instead of proving
$\norminf{\mathbf{X}_{Stern}} < \beta$, we need a proof with different constraints on the witness' components. We
present a solution for this problem.
\subsubsection*{Our construction}
% There are several requirements need to be
% satisfied:
\begin{description}
% \item The Stern protocol leaks the Hamming Weight of the witness, we can cover this
%   leakage by using some kind of padding technique: to make sure that the number of 0s and
%   1s in the message must be equal to each other
% \item we can also apply the decompose technique of \cite{ling2013improved} in
%   each of the elements of the prover witness.
\item [The idea.] is that, instead of
  proving $\norminf{x_i} < \beta_i$, or proving that all the coefficients of $x_i$ are in the range $\{-\beta_i, \dots, \beta_i\}$, it can also be proved that $x_i + \beta_i.f(x)$ is in the range $\{0, \dots, 2\beta_i\}$, where
  $f(x) = 1 + x + x^2 + \dots + x^{n-1}$. Secondly, if we decompose \(x_{i} + \beta_{i}.f(x)\) to its binary
  representation and apply the Stern's variant of \cite{kawachi2008concurrently} to prove the relation

  \begin{equation*}
    \begin{multlined}[][\LTcapwidth]
      R_{KTX} = \{\mathbf{((A,y),x)} \in \mathcal{Z}_{q}^{n \times m} \times \mathcal{Z}_{q}^{n}\times \{0,1\}^{m}:
      wt(\mathbf{x}) \wedge \mathbf{A . x = y} \mod q\},
    \end{multlined}
  \end{equation*} the proof for the original relation \(R_{BV}^{q,n,t,\chi}\) can be obtained. Note that, at this point
  the $Prover$'s witness is a binary vector, that is, if it needs to be to proved that some part of the message is binary, such a proof can be obtained at this point as well. It is important to use such a proof for latticed-based cryptosystems where the
  message space is $\mathbf{R}_2$: a proof for the ISIS relation would be irrelevant in such situations, as it only asserts that the infinity norm of the whole witness is less than some $\beta$.

\item [Protocol description.]
  The protocol \textbf{SternBV(A,y,x)} works as follows: Let $A$ be a matrix of $m \times l$ ring element ($A \in R_{q}^{m \times l}$),
  $\mathbf{x}$ be a vector of $l$ ring elements $\mathbf{x} = \left\{ \mathbf{x_1, x_2, \dots, x_l} \right\}$ and
  similarly $\mathbf{y} = \left\{ \mathbf{y_1,\dots,y_m} \right\}$. The protocol includes the following steps:

\item [Step 1.] Normalizing the bound of each component $x_i$ of \textbf{x} from $\{-\beta_i,\dots,\beta_i\}$ to
  $\{0,\dots,2^{l_i}\}$, where $l_i$ is the smallest integer satisfying $2^{l_i} > (2\beta_i -1)$. This step is done by
  one ring multiplication for each $\mathbf{x_i}$, let $x_i' = x_i + \beta_i.\mathbf{f}(x)$, where
  $f(x) = 1 + x + x^2 + \dots + x^{n-1}$. After this normalization step, instead of proving the relation
  $\mathbf{a_i x_i = y_i}$ with $\norminf{x_i} \in \left\{ -\beta_i, \dots, \beta_i \right\}$, we prove
  $\mathbf{a_i x'_i = y'_i}$ with $\norminf{ x'_i } \in \left\{ 0,\dots,2^{l_i} \right\}$, where
  $\mathbf{y'_i = y_i + a_i\beta_i f(x)}$.

\item [Step 2.] Decompose $x'_i = x_i + \beta_i$ into their binary representation
  \[
    \mathbf{x''_i} = \sum_{j = 0}^{l_i -1}{2_j b_j}
  \]
  Let $\mathbf{x''}$ be the result ring element that concatenates all $\mathbf{x''_i}$ and has $L = \sum{l_i}$
  coefficients. In this step we need to hide the Hamming Weight of the secret vector $\mathbf{x'_i}$. This hiding task
  is achieved by padding:
  \begin{enumerate}
  \item Let $\zeta_0$ and $\zeta_1$ be the number of coefficients of $\mathbf{x''}$ equal to 0 or 1, respectively.
  \item Sample a random vector $\mathbf{\zeta} \in \left\{ 0,1 \right\}^{L}$ that has $(L - \zeta_0)$ coefficients 0 and
    $(L - \zeta_1)$ coefficients 1.
  \item Output $\mathbf{x_{Stern}} = \mathbf{x''} || \mathbf{\zeta}$.
  \end{enumerate}
  The resulting binary vector $\mathbf{x_{Stern}}$ has length $2L$ and the total number of 0s and 1s in the
  $\mathbf{x_{Stern}}$ are the same.

\item [Step 3.] We denote $ rot({\textbf{c}}) \in \mathbb{Z}_Q^{n\times n}$ to be an anti-circulant square matrix, whose
  first column is $\mathbf{c}$, the remaining columns of it being the cyclic rotations of $\mathbf{c}$ with the cycled entries
  negated
  \[
    rot({\mathbf{c}})=
    \begin{bmatrix}
      c_0 & -c_{n-1} & -c_{n-2} & \dots\\
      c_1 & c_0 & -c_{n-1} & \dots\\
      \dots & \dots & \dots & \dots\\
      c_{n-1} & c_{n-2} & c_{n-3} & \dots
    \end{bmatrix}
  \]
  The matrix $A$ can be then reconstructed as $rot$ matrices:
  \[
    \forall \mathbf{a_{i,j}} \in \mathbf{A}: \mathbf{a'_{i,j}} =
    rot(\mathbf{a_{i,j}})
  \]
  The result expanded matrix is denoted $\mathbf{A'}$. We also need to pad the resulting matrix with a corresponding
  number of 0s to make sure that $\mathbf{A'}$ complies with all $x'_i$. Let $\mathbf{A_{Stern}}$ be the padded result.

\item [Step 4.] Modify $\mathbf{y_i}$: Let $\mathbf{y'_i} = \mathbf{y_i + a_i\beta_if(x)}$ and let $\mathbf{y_{Stern}}$ be
  the concatenation of all $\mathbf{y'_i}$.


\item [Step 5.] Run the Stern protocol (Section \ref{append:Stern}) for the proof of $\mathbf{A_{Stern}x_{Stern} = y_{Stern}}$.

\item [Result.] Our protocol has the following properties:
  \begin{itemize}
  \item The knowledge extractor produces different $x_i$ with $\norminf{x_i} \leq \beta_i$. Inheriting from the original
    Stern protocol, the extraction gap is $\gamma = 1$.
  \item The communication cost is $2(n\log q){\sum{l_i}} + commitmentSize$ for each round.
  \end{itemize}

\end{description}

  \begin{algorithm}
    \caption{ZKPoPK Improved for BV}\label{alg:ZKPBVImproved}
    \begin{algorithmic}[1]
      \Procedure{ZKPBV}{$(\mathbf{c}, pk),(\mathbf{m, s, e, e'}))$}
      \State $rot_{c_1} \gets rot(\mathbf{c_1})$
      \State $rot_{p_1} \gets rot(\mathbf{pk_1})$
      \State let I be the $n \times n$ identity matrix
      \State let Z be the $n \times n $ zero matrix
      \State $\mathbf{A} \gets ((rot_{c_1},-tI, Z, -I),(rot_{p_1}, Z, tI,Z)) $
      \State $\mathbf{x} \gets (\mathbf{s, e', e, m})$
      \State $\mathbf{y} \gets (\mathbf{-c_0, -pk_0})$
      \State \textbf{Return} $\mathbf{SternExt(A,x,y)}$
      \EndProcedure
    \end{algorithmic}
  \end{algorithm}


\section{Our Protocol}
\label{sec:our_protocol}
We denote $\user$ to be the client and $\server$ to be the server. There are 3 main submodules in the protocol: Setup,
Enrol, and Authenticate.
\begin{description}
\item[Setup.] $\user$ and $\server$ initialize the parameters, taking into account the following categories:
  \begin{description}
  \item[Biometric Authentication System Parameters.] These parameters are standard ones used by non privacy preserving
    biometric authentication systems:
    \begin{itemize}
    \item False Acceptance Rate (FAR) and False Rejection Rate (FRR)
    \item $a$: The maximum number of incorrect authentication attempts allowed by $\server$.
    \item $\tau$: Threshold to compare the Hamming Distance to decide the authentication result.
      % \item $B$: An integer defines the base of the HD in a
      %   decomposition operation.
    \item \(n'\): The bit-length of the encoded
      biometric data.
    \end{itemize}
  \item[Ring-LWE based techniques parameters.] These parameters are used in the lattice-based cryptosystem which provides
    client privacy against long term quantum attacks.
    \begin{itemize}
    \item $\lambda$: General security parameter of the cryptosystem (the adversary's winning chance in the CPA security game is \(1/2^{\lambda}\))
    \item $n$: Integer $n$ defining the plaintext and ciphertext rings. This will be refered to as the degree of
      the polynomial objects or dimension of the underlying lattice during correctness and security proofs.
    \item $t$: Integer $t$ defining the plaintext space ring $R_t = \mathbb{Z}_t[x]/x^n+1$.
    \item $q$: Integer $q$ defining the ciphertext space ring $R_q = \mathbb{Z}_q[x]/x^n+1$
    \item $\chi_{\alpha q}$: A distribution used to sample noises for LWE-based techniques.  Typically, $\chi$
      is a Gaussian distribution with standard deviation $\alpha q$.
    \item \(\delta\): Renyi Divergence parameter for the security of noise masking.
    \end{itemize}
  \item[Keygen.] Keys are generated for $\user$:
    \begin{itemize}
    \item Secret key: $\mathbf{s} \randomsample \chi_{\alpha q}^n$, \(sk = (1, \mathbf{s, s^{2}, ...})\)
    \item Public key: $pk = \mathbf{(p_0,p_1)}$, with $\mathbf{p_1} \randomsample R_q$ and
      $\mathbf{p_0} = -\mathbf{p_1s} - t\mathbf{e}$, with $\mathbf{e} \randomsample \chi_{\alpha q}^n$.
    \end{itemize}
  \end{description}
\item [Enrolment.] $\user$ extracts the biometric template $\mathbf{x}$, the bit string $\mathbf{x}$ being
  represented as a ring element of ${R}_t$.  The encryption is done by $\enc{\mathbf{x}} = (\mathbf{c_0},\mathbf{c_1})$
  and sent to $\server$.
\item [Authentication.] The following steps are required:
  \begin{enumerate}
  \item $\user$ extracts his biometric features again $\mathbf{y}$ to use them as the query. $\user$ sends
    $\enc{\mathbf{y}} = (\mathbf{c_0', c_1'})$ to $\server$.
  \item ZKP for the first relation: $\user$ has to prove that $\enc{\mathbf{y}}$ is a valid encryption, that is, it
    encrypts a bit string under the BV cryptosystem using the corresponding secret key. This is done by module
    \textbf{SternBV(A, y, x)} described in Sect. \ref{sec:Stern-basedZKP}, where
    $\mathbf{A} = \begin{bmatrix} c_{y0}, t, 0, 1\\p_{y0}, 0, t, 0
    \end{bmatrix}$, \(\mathbf{X} = [\vec{s},\tilde{e_{y}},\vec{e_{0}},y]^T\) and \(\mathbf{Y
    = [c_{y1},p_{y1}]}^{T}\).

  \item HD Computation: $\server$ computes $\enc{HD_{\mathbf{x,y}}}$
    using procedure \ref{sub:ciphertext_packing}. We note that the noise term $\mathbf{e_{HD}}$ of
    $\enc{\mathbf{HD,e_{HD}}}$ can leak information
    about $\mathbf{x}$ when $\mathbf{HD}$ is decrypted. Therefore,
    an extra step is needed to secure the operation.
    \begin{itemize}
    \item Sample $\mathbf{e_{r}} \randomsample \chi_{r}^n$ such that $\norminf{\mathbf{e_r}}$ is big enough compared to
      $\norminf{\mathbf{e_{HD}}}$.(Section \ref{sec:Renyi})
    \item Compute $\enc{\mathbf{r,e_r}}$ and carry out an homomorphic addition operation to mask both the values of
      $\mathbf{HD}$ and the noise $\mathbf{e_{HD}}$:
      $\enc{\mathbf{HD', e'_{HD}}} = \enc{\mathbf{HD, e_{HD}}} + \enc{\mathbf{r,e_r}}$
    \end{itemize}
    The result $\enc{\mathbf{HD'}}$ is then sent to $\user$.
  \item \(\user\) decrypts $\enc{\mathbf{HD'}}$ and derives the actual value ${HD'}$ from the first coefficient of the
    plaintext: \( dec\enc{\mathbf{HD'}} = HD' + r_1 + r_2 + \dots + r_{n-1} \). \(\user\) sends \(HD'\) to \(\server\).
  \item \(\user\) proves that it does the decryption honestly, this is done similarly to step 2.
  \item \(\server\) unmasks \(HD'\) and outputs the authentication result \(HD \stackrel{?}{<} \tau\)
  \end{enumerate}

\end{description}

\subsubsection{Correctness}
\label{sec:correctness}
In the enrolment step, as mentioned before, in order to encrypt $\mathbf{x}$, $\user$ samples
$\mathbf{u,f,g} \randomsample \chi_{\alpha q}^n$ and does $\mathbf{c_0} = \mathbf{p_0u} + t\mathbf{f} + \mathbf{x}$ and
$\mathbf{c_1} = \mathbf{p_1u} + t\mathbf{g}$. The condition for decryption correctness is
$[\langle \mathbf{c_0 + c_1s} \rangle]_q < q/2$, or, $-t\mathbf{eu} + t\mathbf{f} + t\mathbf{gs} < q/2$.

In the authentication step, \(\enc{HD}\) is decryptable if \(\norminf{\langle \enc{HD}, \mathbf{s} \rangle} < q/2\), if
we let \(U\) to be the upper bound of \(\norminf{\langle \enc{HD}, \mathbf{s} \rangle}\), and considering that
\(\norminf{a + b} \leq \norminf{a} + \norminf{b}\) and \(\norminf{a.b} \leq n.\norminf{a}.\norminf{b}\). From theorem
~\ref{theo:HDComputation}, we can derive \(\norminf{\langle \enc{HD},\mathbf{s} \rangle} \leq 2nU + 2nU^{2}\). As in the
work of \cite{naehrig2011can}, we can take \(U\) to be \(2t \sigma^{2} \sqrt{n}\), which is an experimental
estimation. Therefore, the final correctness condition for authentication is \( 16n^{2}t^{2}\sigma^{4} < q\).
\begin{lemma}[Condition for Correct Decryption of HD]
  \label{le:hdcorrectness}
 For the BV encrypted Hamming Distance \(\enc{HD}\), the decryption recovers the correct result if \(\langle \enc{HD},
 \mathbf{s} \rangle\) does not wrap around mod q, namely, if \(16n^{2}t^{2}\sigma^{4} < q\).
\end{lemma}

\subsubsection{Security}
\label{sec:security}
The proposed scheme satisfies the security notions defined in Section \ref{sec:syntaxModel}, proofs are
provided in Appendix \ref{append:Proofs}
\begin{theorem}[Server side security]
  \label{theo:server}
  Under the IND-CPA security of BV cryptosystems, and the zero-knowledge property of the Stern protocol, the proposed
  scheme satisfies (Honest But Curious) Server Privacy Security.
\end{theorem}
\begin{theorem}[Client side security]
  \label{theo:client}
  Under the IND-CPA security of BV cryptosystem and the soundness property of the underlying Stern protocol, the
  proposed scheme satisfies Impersonation Security. Concretely, for $\delta>0$, the protocol is $(q,c)$-secure against
  impersonation with $c \leq c(\delta) + 3 \cdot c_1$, assuming the underlying non-private biometric protocol has
  impersonation probability $\varepsilon_{bio}$ and the underlying Stern ZK protocols have knowledge error
  $\eps_{ZK1},\eps_{ZK2}$ such that
  $q(\varepsilon_{ZK1}+\varepsilon_{ZK2}) + \delta \leq c_1 \cdot
  \varepsilon_{bio}$, $c(\delta) = 2 e^{1+2\delta}$, and the condition $\sigma/r_0 \geq 4 \pi k n q$ holds, with
  $k = 1 + \sqrt{1/\pi \ln(2nq/\delta)}$ and $r_0$ as an upper bound on the size of the noise in $C_{HD}$.
\end{theorem}

\section{Security Proofs}
\label{append:Proofs}
\subsection{Security Proof for Theorem~\ref{theo:server}: Privacy against HBC Server}
\label{append:ProofsPrivacy }

The proof of Theorem \ref{theo:client} can be gained using a sequence of games between a challenger $\challenger$ and
an adversary $\attacker$. We present a sequence of games below. The idea is to remove $sk_k$ and $X_k$ from being
used to compute the view of $\attacker$, except for $Verify(X_k,Y_k^{j})$ queries, as in the ideal game, relying on the
correctness of the protocol and the IND-CPA security of the BV encryption scheme.

\textit{Game 0}. Game 0 is the original real privacy game, in which $(T_k =(pk_k, C_k = Enc^{(1)}_{pk_k}(X_k)))$ is
given by $\challenger$ to $\attacker$ at the beginning of the game. Then, for $j=1\ldots,q$, the attacker sends
$Y_k^{(j)} \in \{0,1\}^n$ to $\challenger$, and the latter simulates a run of the authentication protocol between an
honest client with input $(k,Y_k^{(j)}, sk_k)$ and an honest server with input $(k,T_k)$, returning to $\attacker$ the
protocol's view $V^{(j)}_S$ of the server. Finally, $\attacker$ outputs a bit $\beta$. In the following Game $i$, we let
$S_i$ denote the event that $\beta=1$.

\textit{Game 1}. The computation of the authentication result bit $res^{(j)}$ sent by the server to the client
from the decryption of the ciphertext \(\enc{HD'}\)(its value in Game 0) is changed into the result returned by $Verify(X_k,
Y^{(j)})$. By the correctness constraint of the protocol, this does not change the value of $res^{(j)}$, so $\Pr[S_1] = \Pr[S_0]$.

\textit{Game 2}. The computation of the zero-knowledge protocol transcripts is modified: Instead of computing the
transcripts using the secret witnesses, we simulate them using the statistical zero-knowledge simulator algorithms for
the zero-knowledge proofs. By the zero-knowledge property, this is a perfect simulation, the result of it being $\Pr[S_2]=
\Pr[S_1]$.

\textit{Game 3}. We change the computation of the ciphertexts $C^{(j)}_i$ for $i=0,\ldots,l-1$ and $Q^{(j)}_k$ for
$j=1,\ldots,q$ and $C_k$ to encrypt zero messages, instead of encrypting the secret-related messages resulting from the previous
game. Since $sk_k$ is thus not used any longer for generating the view of $\attacker$, it follows by a hybrid argument that
$|\Pr[S_3]-\Pr[S_2]| ((l+1) \cdot q + 1) \cdot \varepsilon_{BV}$. $\varepsilon_{BV}$ denotes the maximal advantage
of an attacker against IND-CPA of the BV scheme against attacks with run-time $T + \mathrm{poly}(n, \log Q)$, where $T$ is
the run-time of $\attacker$.  In this game, since the only information on $X_k$ comes via the $Verify(X_k,Y_k^{j})$
queries, the challenger together with $\attacker$ constitute an efficient attacker against the ideal privacy game, which
outputs 1 with a degree of probability, different by at most $(l+1) \cdot q + 1) \cdot \varepsilon_{BV}$ from the probability of
outputting 1 in the real privacy game, as required.

\subsection{Security Proof for Theorem~\ref{theo:client}: Type I Impersonation attack}
The proof of theorem \ref{theo:server} and \ref{theo:client} can be obtained using a sequence of games between the
challenger $\challenger$ and the adversary $\attacker$. We here present such a sequence, as well as the relations among
them, to demonstrate the type I security model proof.

\textit{Game 0}. Game 0 is the original impersonation game for type I attack.
\begin{description}
\item [Setup.] $\challenger$ intiates $D_k \randomsample D_{bio}$ and $X_k \randomsample D_k$. $\challenger$ sets up
  $(sk_k, pk_k)$ and launches $Enrol(k, X_k)$ to get $(sk_k, T_k =(pk_k, C_k = Enc^{(1)}_{pk_k}(X_k)))$.  $\attacker$
  submits a type 1 attack query and receives $sk_k$ from $\challenger$.
\item [Query.] $\attacker$ runs $q$ authentication sessions. In each session $j = 1, \dots, q$, $\attacker$ sends
  $(Q_k^{(j)} = Enc^{(1)}(Y^{(j)}))$. $\attacker$ and $\challenger$ run
  $\mathbf{ZKPValidEnc}(Q_k^{(j)},Y^{(j)})$. $\challenger$ evaluates
  $C_{HD} = \mathbf{EvalDistance}_{pk_k}(C_k, Q_k^{(j)})$ then computes
  $C_{HD'} = (-1)C_{HD} + Enc_{pk_k}^{(1)}(r_{HD'},e_{HD'})$ for $r_{HD'} \randomsample \mathcal{P}$ and
  $e_{HD'} \randomsample \chi_{HD}$. The resulting ciphertext is sent to $\attacker$, $\attacker$ decrypts $C_{HD'}$, 
  decomposes into bits and sends back the ciphertexts
  $C_0^{(j)}, \dots, C_{l-1}^{(j)} (= Enc^{(1)}(b_i), i = 0,\dots, l-1)$ as well as the re-encryption
  $C_{HD_0'}$. $\attacker$ and $\challenger$ run $\mathbf{ZKPUnpack}(C_{HD},C_{HD_0'})$ and
  $\mathbf{ZKPBinDecomp}(C_{HD_0'}, C_i^{(j)})$.  $\challenger$ evaluates
  $C_{HD}'' = Enc^{(2)}(2^l + t) + \mathbf{ToUnary}(C_i^{(j)}) + Enc^{(2)}(-r) $ to get $Enc^{(2)}(2^l + t -
  HD)$. $\challenger$ calculates $Enc^{(3)}(res) \gets \mathbf{MSBExtract}(C_{HD}'')$ and sends the result to $\attacker$. $\attacker$
  decrypts it and sends the authentication result bit back. $\challenger$ and $\attacker$ perform
  $\mathbf{ZKPCorrectDec}(res)$ to assess the authentication result. At the end, the server outputs \textbf{Accept} if
  all the proofs pass and $res = \mathbf{Accept}$.
\end{description}
Next, we discuss following games, the plan being to proceed towards the final game, where everything related to $X_k$
$\attacker$ receives can be simulated without any knowledge about $X_k$ (the only known information about $X_k$ is
the function $Verify(X_k, Y)$). Let $res_s = Verify(X_k, Y^{(j)})$ and $S_i$ be
the event in the game $i$ such that $res_s = Accept$.\\
\textit{Game 1}. In this game, we abort $\mathbf{ZKPValidEnc}$ if $Q_k^{(j)}$ is not a valid encryption of the query,
but the $Prover$ manages to pass the proof. Let $(*)_1$ be this event.
\[
  (*_1)res_s = \begin{cases}
    \text{Reject if } \mathbf{ZKPValidEnc} \text{ fails}\\
    \text{res else}
  \end{cases}
\]
Let $bad_0$ be the event in game 0 and $\exists j \leq q\ s.t\ (*_1)\ \textnormal{be the case}$. We want to show that when we
modify \textit{game 0}, the probability of a successful forgery $S_1$ in \textit{game 1} is not much lower than before. Due to the modification, we observe that $ Pr[S_1] \geq Pr[S_0] - Pr[bad_0] $. For any $j$ in the $q$
authentication attemps, by the $\varepsilon-soundness$ property of $\mathbf{ZKPValidEnc}$ as a proof of membership in
(*), $Pr[(*_1)\ occurs\ for\ some \ j] \leq \varepsilon_{ZK1}$ holds. That is, the probability of $bad_0$ would be the union
of these events, bounded by $Pr[bad_0] \leq q\varepsilon_{ZK1}$. In other words, the advantage of $\attacker$ in
\textit{game 1} is
\[
  Pr[S_1] \geq Pr[S_0] - Pr[bad_0] \geq \varepsilon_{imp} - q\varepsilon_{ZK1}
\]
\\
\textit{Game 2}. In this game, we abort $\mathbf{ZKPUnpack}$ if in one of the $j^{th}$ runs, the $Verifier$ accepts but
the ciphertext does not satisfy the relation. Let $bad_1$ be this event. By the same type of argument, we can derive
$Pr[bad_1] \leq q\varepsilon_{ZK2}$ and, therefore,
$$Pr[S_2] \geq Pr[S_1]
- Pr[bad_1] \geq \varepsilon_{imp} - q(\varepsilon_{ZK1}+\varepsilon_{ZK2})
$$
\textit{Game 3 and Game 4}. Similarly, we abort $\mathbf{ZKPBinDecomp}$ and $\mathbf{ZKPCorrectDec}$ if in any $j^{th}$
runs, the $Verifier$ accepts results even when the correctness constraint of the ZKP is not fulfilled. We thus obtain
$$Pr[S_3] \geq  \varepsilon_{imp} - q(\varepsilon_{ZK1}+\varepsilon_{ZK2}
+\varepsilon_{ZK3})
$$ and
$$Pr[S_4] \geq  \varepsilon_{imp} - q(\varepsilon_{ZK1}+\varepsilon_{ZK2} +
\varepsilon_{ZK3} + \varepsilon_{ZK4})
$$
By the end of \textit{Game 4}, if in any authentication attempt $j$, none of the 4 games have been aborted, then, by
the correctness property of ZKP, the server's output is equal to the output of $Verify(X_k, Y^{(j)})$. In other words, we
have shown what we could get by just querying the oracle $Verify()$. Next, we want to simulate everything related to
$X_k$ the attacker can see using
just that oracle.\\
\textit{Game 5.} At the end of \textit{Game 4}, $\challenger$ has the bit $b = Verify(Y^{(i)},X_k)$. In this game, we
can change $C_{res}^{(j)}$ from $\mathbf{MSBExtract}(C_{HD}'') = Enc(b;e_{res})$ to $Enc(verify(Y^{(i)},X_k);e_{res})$,
given that the challenger can use the secret key to extract $e_{res}$. By doing this change, $\attacker$ still sees the
same ciphertext, so the probability of winning of $\attacker$ in this game
is the same as in \textit{Game 4}.\\
\textit{Game 6.} In this game, we change the way $\challenger$ computes $C_{HD}''$: In the orignal game 0, $Enc(-r)$ was
added to remove the mask. We want to prevent this $r$ from being used anywhere in the game, so we replace $Enc(-r)$ by
$Enc(0)$. This change does not affect $\attacker$'s success probability: $r$ only affects the plaintext inside
$C_{HD}''$, since we do not use this plaintext anymore (it has been replaced in \textit{Game 5}), this change therefore
does not affect what the attacker sees. Again, $Pr[S_6] = Pr[S_5] = Pr[S_4]$.\\
\textit{Game 7.} We modify the way $C_{HD}'$ is computed in this game. Instead of calculating
$C_{HD}' \gets (-1)C_{HD} + Enc(r;e_{HD'})$, the challenger chooses a random $HD' \randomsample \mathbb{Z}_p$ and
encrypts it with the noise used before: $C_{HD}' \gets Enc(HD'; -e_{HD} + e_{HD'})$. In this game, the plaintext has
changed from being $r + HD$ to a uniform $HD' \in \mathbb{Z}_p$.  Since $r$ is also uniform in $\mathbb{Z}_p$, the
attacker sees a uniform
plaintext in both cases. Therefore, $Pr[S_7] = Pr[S_6]$.\\
\textit{Game 8.} Finally, we set $C_{HD'} = Enc(HD', e_{HD'})$ to $e_{HD'} \randomsample \chi_{mask}$ instead of
$-e_{HD} + e_{HD'}$. We replace the sum of the Gaussian noise with random noise. In Section \ref{sec:Renyi}, we
showed that $Pr[S_8] \geq \frac{1}{c(\delta)}(Pr[S_7]-q \cdot \delta)$, where
$c(\delta) = RD(-e_{HD} + e_{HD'}, e_{HD}) \leq 2 \cdot e^{1+2\delta}$ by Lemma~\ref{le:Renyi} and by our assumption about the value of the
parameters. After we finish \textit{Game 8}, we can see that all the messages the attacker sees, can be simulated
using only the verified bit $b = Verify(Y^{(i)}, X_k)$. We now have an attacker $A'$ against the biometric impersonation
with advantage:
\[
\varepsilon_{bio} = Adv(A') = Pr[S_8] \geq \frac{1}{c(\delta)}(\varepsilon_{imp} - q(\varepsilon_{ZK1}+\varepsilon_{ZK2} +
\varepsilon_{ZK3} + \varepsilon_{ZK4}) + \delta),
\]
which results in the claimed bound $\varepsilon_{imp} \leq c(\delta) \cdot \varepsilon_{bio} + q \cdot (\varepsilon_{ZK1}+\varepsilon_{ZK2} +
\varepsilon_{ZK3} + \varepsilon_{ZK4}) + \delta \leq (c(\delta)+c_1) \cdot \varepsilon_{bio}$ if $q(\varepsilon_{ZK1}+\varepsilon_{ZK2} +
\varepsilon_{ZK3} + \varepsilon_{ZK4}) + \delta \leq c_1 \cdot \varepsilon_{bio}$.


\subsection{Security Proof for Theorem~\ref{theo:client}: Type II Impersonation attack}
\label{append:ProofsTypeII}
The proof of Theorem \ref{theo:client} can be met using a sequence of games involving the challenger $\challenger$ and the adversary $\attacker$. We hereafter present
a sequence of games as well as the relations among them to demonstrate the type II security model proof. The idea is that in this type of attack, $sk_k$ is not used to compute the view of $\attacker$. On the other hand, the soundness of the zero-knowledge proof $\mathbf{ZKPValidEnc}$ implies the existence of an efficient witness extractor algorithm, capable of extracting the witness (i.e. the secret key $sk_k$) from a cheating prover able to succed with a significantly higher probability than the knowledge error of the zero-knowledge proof, which contradicts the IND-CPA security of the BV encryption scheme. \\\\
\textit{Game 0}. Game 0 is the original impersonation game for type II attack, i.e. the same as Game 0 in the proof of security against Type I attacks, except that $(T_k =(pk_k, C_k = Enc^{(1)}_{pk_k}(X_k)))$ is given by $\challenger$ to $\attacker$ at the beginning of the game, rather than $sk_k$.
For $j \in \{1,\ldots,q\}$, let $res^{j}$ denote the result of the $j$th authentication protocol run between $\attacker$ and $\challenger$, and $S_0$ be
the event in the game $0$ such that $res^{j} = Accept$ for some $j=1,\ldots,q$. We have $\Pr[S_0] = \varepsilon_{imp,II}$, the type II success probability of $\attacker$. \\\\
\textit{Game 1}. From the definition of event $S_0$ in Game 0, it follows that there exists some $j^* \in \{1,\ldots,q\}$ such that $\Pr[res^{j^*} = Accept] \geq \varepsilon_{imp,II}/q$. Furthermore, by an averaging argument, there must exist a set $G$ of $(D_k,X_k,pk_k)$ such that $\Pr[(D_k,X_k,pk_k) \in G] \geq \varepsilon_{imp,II}/(2 \cdot q)$, and for each $(D'_k,X'_k,pk'_k) \in G$, we have $\Pr[res^{j^*} = Accept |(D_k,X_k,pk_k)=(D'_k,X'_k,pk'_k)] \geq  \varepsilon_{imp,II}/(2 \cdot q)$. By $\varepsilon_{ZK1}$-soundness of the zero knowledge proof $\mathbf{ZKPValidEnc}$ (cite Golderich's `Foundations of Cryptography' book, volume 1, Prop. 4.7.5), there exists a witness extractor algorithm that runs in expected time $T'=O(\mathrm{poly}(n \log Q) \cdot T / (\varepsilon_{imp,II}/(2*q) - \varepsilon_{ZK1}))$, where $T$ denotes the run-time of $\attacker$ and outputs a witness containing $sk_k$ for ${ZKPValidEnc}$. Therefore, we obtain a (secret key recovery) attack algorithm against the IND-CPA security of the BV encryption scheme with the expected run-time $T'$ and advantage $\varepsilon' \geq \varepsilon_{imp,II}/(2 \cdot q)$. Hence,  if $\varepsilon_{imp,II}/(2*q) - \varepsilon_{ZK1} > \varepsilon_{imp,II}/(4*q)$, or equivalently, if $\varepsilon_{imp,II} > 2 \cdot q \varepsilon_{ZK1}$, then we obtain a contradiction with the assumption of the BV encryption scheme with parameters $Q,n,\sigma$ is IND-CPA against attacks with expected time $O(\mathrm{poly}(n \log Q) \cdot T / \varepsilon_{ZK1})$ and advantage $\geq \varepsilon_{ZK1}$. It follows under the latter assumption that $\varepsilon_{imp,II} \leq 2q \varepsilon_{ZK1} \leq 2 c_1  \cdot \varepsilon_{bio}$, under the assumption that  $q(\varepsilon_{ZK1}+\varepsilon_{ZK2} +
\varepsilon_{ZK3} + \varepsilon_{ZK4} + \delta) \leq c_1 \cdot \varepsilon_{bio}$.


\section{Result Evaluation}

\subsection{Parameters}
\label{sec:parameters}
We now consider how to set concrete parameters. From Lemma \ref{le:hdcorrectness} and \ref{le:Renyi} and
\cite{naehrig2011can}, we can take \(\sigma = 50\) to make the protocol secure against the lattice attacks
(\cite{micciancio2008lattice}) and
also satisfy the circuit privacy requirement. We can take \(t = 2048\) to define the plaintext space ring \(R_{t}\),
this is enough to cover Hamming Distance of bit strings up to 2048 bits. We need \(n \geq 2048\) for the packing
ciphertext. With these parameters, we need \(q\) to be approximately 70 bits to satisfy \(16n^{2}t^{2}\sigma^{4} < q\).
The proposed scheme works with only 1 level of homomorphic multiplication for the proposed set of parameters
$(n = 2048, q \approx 2^{70}, t = 2048, \sigma = 50)$, proof of concept implementation can be found in github
\cite{rimrimrq38:online}, communication size for ZKP is approximately 8MB to provide security against a malicious client
model.

\subsection{Limitations and open problems}
\label{sec:limitations}
We expose the HD to the server in the last step of the protocol and let the server do the threshold comparison
operation in the plaintext domain. We believe that, given such a value, the server should not be able to learn any
information about the original bit strings template (assuming the provable CPA-secure ciphertexts of the underlying
cryptosystem, and the HD noise is likely independent of templates). Doing the comparision of HD with threshold
homomorphhically proves much less efficient, and removing this assumption is left as an open problem.

Also, we assume an honest but curious server, which is reasonable against passive exposure attacks. Active attacks are
harder to carry out undetected and slower, provided the server can be audited regularly. We emphasize that previous quantum
resistant protocols also made this assumption, and were not even able to defend against passive honest but curious trusted
parties. Defending privacy against a malicious server is left as an open problem.

Finally, the communication size of the Stern-based ZKP protocol is large, due to the round soundness error \(2/3\) (many
communication rounds will be needed for security). We leave the problem of how to reduce such overhead for future works.



\section{Conclusion}
\label{sec:firstProcResults}




%%% Local Variables:
%%% mode: latex
%%% TeX-master: "../thesis"
%%% End:
