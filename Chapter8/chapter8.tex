\chapter{Conclusion and Open Problems}
\label{chap:conclusion}

% **************************** Define Graphics Path **************************
\ifpdf
    \graphicspath{{Chapter8/Figs/Raster/}{Chapter8/Figs/PDF/}{Chapter8/Figs/}}
\else
    \graphicspath{{Chapter8/Figs/Vector/}{Chapter8/Figs/}}
\fi

% ***** Main ****

Throughout this thesis, we have developed different solutions to privacy
preserving biometrics authentication. Starting from a protocol with an inspiring
technique of ciphertext packing but lacking of protection against a malicious
client model, we constructed a Zero Knowledge Proof technique to provide a
solution to the problem. The technique also causes the protocol not to rely on any
trusted third party to keep the client's secret key for decrypting results from
homomorphic operations. Discussing the next protocol, we noted that, to fully support
multi-factor security, especially in the biometrics authentication context, the
circuit privacy problem needs to be solved efficiently. We proposed to enforce the Renyi
Divergence technique in the security analysis, so that the protocol can preserve
its security level while still keeping the parameter settings within practical
thresholds. Specifically, we showed that the noise within the results of homomorphic
operations can be covered by not-too-large masks to add extra security to the
ciphertext result. The third variant was our first attempt to do all the
essential operations homomorphically: besides computing the Hamming
Distance, we also wanted to compare the ciphertext result with some threshold
value, under the malicious client mode hypothesisl. Although the results were not practical
enough in terms of communication size, we learned different techniques to
balance the tradeoffs between communication and computation. In our last
version, we proposed another alternative to address all the weakness of the
previous protocols. The solution includes the combinations of many advanced
cryptographic techniques: Homomorphic Encryption, Zero Knowledge Proof, Garbled
Circuit and Oblivious Transfer. However, we still notice a significant one time
setup cost.

Some of the techniques developed in this thesis can be used to construct other
privacy-preserving protocols. For example, the decomposition technique used to
construct Stern-based ZKP can be applied to contexts where balancing
communication/computation is neccessary, or the Renyi Divergence security
analysis technique can be applied to lattice-based cryptosystems requiring
circuit privacy and keeping the parameters within practical thresholds. The
new challenge-response and Oblivious Transfer submodules used in the last
protocol can be used in different Homomorphic Cryptosystems in various contexts.

There are open questions with respect to the schemes we have put forward:
\begin{itemize}
\item We obtain security against a malicious client model and assume the server is
  \textit{semi-honest}. Designing a practical scheme that is also secure against a
  malicious server is an interesting open challenge.
\item We mainly used Hamming Distance to measure the difference between the
  registered and queried templates. There are many other approaches
  \cite{jain201650} yielding a better False Acceptance Rate and False
  Rejection Rate. The applicability of our technique to such methods is left for
  future work
\item The operations of computing and comparing different types of distances can be
applied to contexts unrelated to biometric authentication. 
\item Further optimizations to reduce computation time or communication size, such as the composition of the Zero Knowledge Proof (AND, OR, EQ) can be investigated to improve the practical implementation of protocols.
\end{itemize}

%%% Local Variables:
%%% mode: latex
%%% TeX-master: "../thesis"
%%% End:
