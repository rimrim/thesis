\chapter{Conclusion and Open Problems}
\label{chap:conclusion}

% **************************** Define Graphics Path **************************
\ifpdf
    \graphicspath{{Chapter8/Figs/Raster/}{Chapter8/Figs/PDF/}{Chapter8/Figs/}}
\else
    \graphicspath{{Chapter8/Figs/Vector/}{Chapter8/Figs/}}
\fi

% ***** Main ****

Throughout the thesis, we have developed different solutions to privacy
preserving biometrics authentication. Starting from a protocol with an inspiring
technique of ciphertext packing but lacking of protection against malicious
client model, we constructed a Zero Knowledge Proof technique to provide
solution to the problem. The technique also help the protocol not to rely on any
trusted third party to keep the client's secret key for decrypting results from
homomorphic operations. In the next protocol, we discover that to fully support
multi-factor security, especially in the biometrics authentication context, the
circuit privacy problem needs to be solved efficiently. We proposed Renyi
Divergence technique in the security analysis so that the protocol can preserve
its security level while still keeping the parameter settings within practical
threshold, specifically, we showed that the noise of results from homomorphic
operations can be covered by not-too-large masks to add extra security to the
ciphertext result. The third variant was our first attempt to do all the
essential operations homomorphically: in addition to computing the Hamming
Distance, we also wanted to compare the ciphertext result with some threshold
value, under the malicious client model. Although the results were not practical
enough in terms of communication size, we learned different techniques to
balance the tradeoffs between communication and computation. In our last
version, we proposed another alternative that cover all the weakness of the
previous protocol. The solution include combinations of many advanced
cryptographic techniques: Homomorphic Encryption, Zero Knowledge Proof, Garbled
Circuit and Oblivious Transfer. However, we still notice a significant one time
setup cost.

Some of the techniques developed in this thesis can be used to construct other
privacy-preserving protocol. For example, the decomposition technique used to
construct Stern-based ZKP can be applied to contexts where balancing
communication/computation is neccessary, or the Renyi Divergence security
analysis technique can be applied to lattice-based cryptosystems that required
circuit privacy and keeping the parameters to be in practical thresholds. The
new challenge-response and Oblivious Transfer submodules used in the last
protocol can be used in different Homomorphic Cryptosystems in other context.

There are open questions with respect to the schemes we have constructed
\begin{itemize}
\item We obtain security against malicious client model and assume the server is
  \textit{semi-honest}, designing a practical scheme that is also secure against
  malicious server is an interesting open question.
\item We mainly used Hamming Distance to measure the difference between the
  registered and queried templates. There are lots of other approach
  \cite{jain201650} that promise better False Acceptance Rate and False
  Rejection Rate. The applicability of our techniques to such methods is left as
  future works
\end{itemize}

%%% Local Variables:
%%% mode: latex
%%% TeX-master: "../thesis"
%%% End:
